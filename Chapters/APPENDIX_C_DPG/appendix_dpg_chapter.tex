%
%!TEX root = ../../hp3D_user_guide.tex
%

\chapter{Implementation of the DPG Method}
\label{chap:dpg}

%--------------------------------------------------------------------

The \hp3D library has been the main tool for the numerical studies with the discontinuous Petrov--Galerkin (DPG) Method with Optimal Test Functions \cite{demkowicz2017dpg}. Many of the recent applications implemented in \hp3D (see Appendix~\ref{chap:applications}) were discretized with the DPG method. DPG implementations are straightforward to implement within the \hp3D framework, and we briefly review (in abstract form) how the DPG linear system is constructed. 

Examples of model problem implementations with DPG are also provided (e.g.~Appendix~\ref{sec:poisson-primal}).

\paragraph{The DPG mixed formulation.} Consider an abstract variational problem and its Petrov--Galerkin (PG) discretization:
\[
\left\{
\begin{array}{lllll}
u \in \trial \\
b(u,v) = l(v) \quad v \in \test
\end{array}
\right.
\qquad
\longrightarrow
\qquad
\left\{
\begin{array}{lllll}
u_h \in \trial_h \subset \trial \\
b(u_h,v_h) = l(v_h) \quad v_h \in \test_h \subset \test
\end{array}
\right.
\]
The {\em Petrov--Galerkin } discretization of the problem requires $\dim \trial_h = \dim \test_h$. For $\trial=\test$, the choice of $\trial_h=\test_h$ leads to the {\em Bubnov--Galerkin} (BG) method.
In the {\em Petrov--Galerkin Method with Optimal Test Functions}, we embed the original problem into an equivalent {\em mixed problem}:
\[
\arraycolsep=2pt
\left\{
\begin{array}{lllll}
\psi \in \test, u \in \trial \\
(\psi,v)_\test + b(u,v) & = l(v)  \quad &\ v \in \test\\
b(w,\psi) & = 0 &\ w \in \trial
\end{array}
\right.
\qquad
\longrightarrow
\qquad
\left\{
\begin{array}{lllll}
\psi_h \in \test_h ,\, u_h \in \trial_h \\
(\psi_h,v_h)_\test + b(u_h,v_h) & = l(v_h) \quad &\ v_h \in \test_h \\
b(w_h, \psi_h) & = 0 &\ w_h \in \trial_h
\end{array}
\right.
\]
The mixed formulation involves an additional unknown---$\psi \in \test$---the Riesz representation of the residual. On the continuous level $\psi = 0$ but its discretization $\psi_h \in \test_h$ is different from zero. The test inner product $(\cdot,\cdot)_\test$ enters directly the problem and affects the corresponding FE solution $u_h$. The main motivation for solving the more expensive mixed problem comes from stability considerations: the discrete spaces $\trial_h,\test_h$ need not longer be of the same dimension, and it is easier to satisfy the \emph{discrete inf--sup condition} by simply employing a larger discrete test space. We are solving effectively an overdetermined discrete system.

An alternative is to test with a larger space of {\em broken} test functions $V(\Omega_h) \supset \test$ where $\Omega_h$ denotes the 
decomposition of the domain into finite elements. Testing from a larger space of broken test functions necessitates the introduction of an 
additional unknown---the {\em trace} $\hat{u}$---a Lagrange multiplier. The modified problem looks as follows:
\[
\left\{
\begin{array}{llllll}
u \in \trial ,\, \hat{u} \in \hat{\trial} \\
\underbrace{b(u,v) + \langle \hat{u}, v \rangle_{\Gamma_h}}_{b_{\text{mod}}((u,\hat{u}),v)} = l(v) \quad v \in \test(\Omega_h)\, .
\end{array}
\right.
\]
The new group variable includes the original unknown and the trace unknown. The corresponding PG scheme with optimal test functions is known as the DPG method. The word {\em discontinuous} in the name refers to the use of broken (discontinuous) test spaces.
\[
\arraycolsep=2pt
\left\{
\begin{array}{llllll}
u \in \trial ,\, \hat{u} \in \hat{\trial}, \psi \in \test(\Omega_h) \\
(\psi,v)_{\test(\Omega_h)} + b(u,v) + \langle \hat{u}, v \rangle_{\Gamma_h} & = l(v) \quad &\quad v \in \test(\Omega_h)\\
b(w, \psi) & = 0 &\quad w \in \trial\\
\lb \hat{w}, v \rb_{\Gamma_h} & = 0 &\quad \hat{w} \in \hat{\trial} \, .
\end{array}
\right.
\]

The main advantage of using the broken test space in context of the PG method with optimal test functions is that the Gram matrix resulting from the discretization of
the test inner product $(\psi,v)_\test$ becomes block-diagonal, and the additional variable $\psi \in V(\Omega_h)$ can be statically condensed on the
element level. The whole ``DPG magic'' happens only in the element routine, the remaining part of the code remains the same as for the standard Bubnov--Galerkin method. The mixed problem,
\[
\left(
\begin{array}{ccc}
G & B & \hat{B} \\
B^* & 0& 0\\
\hat{B}^* & 0& 0
\end{array}
\right)
\, 
\left(
\begin{array}{c}
\psi\\u\\\hat{u}
\end{array}
\right)
\, = \,
\left(
\begin{array}{c}
l \\0 \\0
\end{array}
\right)\, ,
\]
becomes a standard Bubnov--Galerkin problem:
\[
\left(
\begin{array}{cc}
B^* G^{-1} B & B^* G^{-1} \hat{B} \\
\hat{B}^* G^{-1} B^* & \hat{B}^* G^{-1} \hat{B}^*
\end{array}
\right)
\, 
\left(
\begin{array}{c}
u\\\hat{u}
\end{array}
\right)
\, = \,
\left(
\begin{array}{c}
B^* G^{-1}  l \\\hat{B}^* G^{-1} l
\end{array}
\right)\, .
\]

Recall that the trace unknown $\hat u$ is discretized in \hp3D by using element-wise restrictions of $H^1$-, $H(\tcurl)$-, and $H(\tdiv)$-conforming elements. This is accomplished by first defining a standard $H^1$, $H(\tcurl)$, or $H(\tdiv)$ physics variable in the \file{physics} input file and then declaring it a trace via the global flag array \var{PHYSAi(:)}. For more information on the traces, see Section~\ref{sec:traces}.

\paragraph{Constructing the DPG linear system.}
In practice, the local element matrices of the DPG linear system can be constructed in the following way:

First, reduce the broken mixed formulation to a matrix equation:
\begin{itemize}
\item{
Let $\mf{U}_h = \{ \mf{u}_i \}_{i=1}^N,\ \mf{\hat U}_h = \{ \mf{\hat u}_i \}_{i=1}^{\hat N}$, and $\mf{V}^r = \{ \mf{v}_i \}_{i=1}^M$ (where $M > N+\hat N$) denote bases for the discrete trial space $\trial_h \times \hat \trial_h$ and the enriched test space $V^r$, respectively.}
\item{
Define the stiffness matrix for the modified bilinear form, the Gram matrix, and the load vector:
\[
	\mr{B}_{ij} = b(\mf{u}_j, \mf{v}_i), \quad 
	\mr{\hat B}_{ij} = \lb \mf{\hat u}_j, \mf{v}_i \rb, \quad 
	\mr{G}_{ij} = (\mf{v}_j, \mf{v}_i)_V, \quad 
	\mr l_i = l(\mf{v}_i) .
\]
}
\item{
In matrix form, the problem can now be formulated in the following way:

Find the set of coefficients (over field $\bb{F} = \bb{R}$ (or $\bb{C}$))
\[
	\mr w = [\mr{w}_i]_{i=1}^N \in \bb{F}^N, \quad
	\mr{\hat w} \in [\mr{\hat w}_i]_{i=1}^{\hat N} \in \bb{F}^{\hat N}, \text{ and }\
	\mr{q} = [\mr{q}_i]_{i=1}^M \in \bb{F}^M
\]
such that
\[
	\mr {u_h} = \sum_{i=1}^{N} \mr{w}_i \mf{u}_i , \quad
	\mr{\hat{u}_h} = \sum_{i=1}^{\hat N} \mr{\hat w}_i \mf{\hat u}_i , \text{ and }\
	\mr{\Psi} = \sum_{i=1}^{M} \mr{q}_i \mf{v}_i
\]
satisfy
\[
	\left[ \begin{array}{ccc}
		\mr G & \ \mr B \ \ & \mr{\hat B} \\
		\mr B^* & \ \mr 0 \ \ & \mr 0\\
		\mr{\hat B^*} & \ \mr 0 \ \ & \mr 0
	\end{array} \right]
	\left[ \begin{array}{c}
		\mr{\Psi} \\
		\mr{u_h} \\
		\mr{\hat u_h}
	\end{array} \right]
	=
	\left[ \begin{array}{c}
		\mr l \\
		\mr 0 \\
		\mr 0
	\end{array} \right] .
\]
}
\end{itemize}

Let $\mr{B} := [\mr{B}\ \mr{\hat B}]$. Then, \vskip -30pt
\begin{align*}
	\mr{B^*} \mr{G^{-1}} \mr{B} &= 
	\mr{B^*} \mr{(L L^*)^{-1}} \mr{B} =
	( \mr{L^{-1}} \mr{B} )^* \overbrace{( \mr{L^{-1}} \mr{B} )}^{\mr{\tilde B}} =
	\mr{\tilde B}^* \mr{\tilde B} \, ,
	\\[3pt]
	\mr{B^*} \mr{G^{-1}} \mr{l} &=
	( \mr{L^{-1}} \mr{B} )^* ( \mr{L^{-1} l} ) =
	\mr{\tilde B}^* ( \mr{L^{-1} l} ) \, .
\end{align*}
That is, the statically condensed DPG linear system is computed as follows:
\begin{enumerate}
	\itemsep 0pt
	\item Cholesky factorization: \hskip 7pt 
	$\mr{G = L L^*}$
	\item Solve triangular system: \hskip 2pt 
	$\mr{[\tilde B \, | \, \tilde l \,] = L^{-1} [B \, | \, l \,]}$
	\item Matrix multiply: \hskip 37pt 
	$\mr{\tilde B^* [\tilde B \, | \, \tilde l \,]}$
\end{enumerate}
The \routine{elem} routine concludes by storing this statically condensed system into the corresponding blocks of the assembly arrays \var{ALOC} and \var{BLOC} (cf.~\ref{sec:coupled-variables}).

%\section*{Historical Comments}




