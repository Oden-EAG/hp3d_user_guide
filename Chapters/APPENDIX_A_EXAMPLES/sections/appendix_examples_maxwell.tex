%
%!TEX root = ../../../hp3D_user_guide.tex
%

%--------------------------------------------------------------------

\section{Maxwell Problems}
\label{sec:maxwell}

For time-harmonic Maxwell problems, the solution is complex-valued; the \hp3D library must therefore be compiled with preprocessing flag \code{\var{COMPLEX}=1}.

The general structure of the \file{MAXWELL} application directory is similar to the implementation of previously discussed applications. This section only focuses on the specific aspects of implementing particular formulations of the Maxwell problem. We encourage reading Section~\ref{sec:poisson} on Poisson problems for additional discussion of problem implementations, in general.


\begin{itemize}
\item
{
Linear time-harmonic Maxwell equations:
\begin{alignat*}{3}
	\curl \bs E + i \omega \mu \bs H
	&= \bs 0 &\quad \text{in } \Omega \, , \\
	\curl \bs H - (i \omega \eps + \sigma) \bs E 
	&= \bs J^{\text{imp}} &\quad \text{in } \Omega \, , \\
	\bs n \times \bs E &= \bs n \times \bs E_0 &\quad \text{on } \Gamma \, .
\end{alignat*}
}
\item
{
Curl--curl formulation:
\begin{alignat*}{3}
	\curl (\mu^{-1} \curl \bs E) - (\omega^2 \eps - i \omega \sigma) \bs E
	&= -i \omega \bs J^{\text{imp}}  &\quad \text{in } \Omega \, , \\
	\bs n \times \bs E &= \bs n \times \bs E_0 &\quad \text{on } \Gamma \, .
\end{alignat*}
}
\item
{
Classical variational formulation:
\[
\left\{
\begin{array}{lll}
	\bs E \in \Hcurl : \bs n \times \bs E = \bs n \times \bs E_0 \text{ on } \Gamma \, , \\[5pt]
	(\mu^{-1} \curl \bs E,\curl \bs F) - ((\omega^2 \eps - i \omega \sigma) \bs E, \bs F)
	= -i \omega (\bs J^{\text{imp}}, \bs F) \, , \\[5pt]
	\hfill
	\quad \bs F \in \Hcurl \, : \, \bs n \times \bs F = \bs 0 \text{ on } \Gamma \, .
\end{array}
\right.
\]
The formulation involves just one unknown $\bs E \in \Hcurl$, defined on the whole domain.
}
\end{itemize}

\subsection{Galerkin implementation}
\label{subsec:maxwell-galerkin}

The Galerkin FE implementation of the classical (curl-curl) variational problem is located in the application directory \file{problems/MAXWELL/GALERKIN}. In the remainder of this section, file paths may be given as relative paths within the application directory.

The physics input file defines the vector-valued unknown: electric field $\bs E \in \Hcurl$.
\begin{lstlisting}[caption=\file{MAXWELL/GALERKIN/input/physics} input file.]
100000              MAXNODS, nodes anticipated
1                   NR_PHYSA, physics attributes
field  tangen  1    H(curl) variable
\end{lstlisting}

The curl-curl formulation above specifies a non-homogeneous Dirichlet boundary condition for the tangential component of the electric field. When setting a Dirichlet flag for an $H(\tcurl)$-variable in the \routine{set\_initial\_mesh} routine, the code automatically interprets the BC to be applied to the tangential component. As in previous examples, the values of the field and its derivative on the boundary must be supplied by the user in the \routine{dirichlet} routine.

The \routine{elem} routine for the Galerkin implementation of the Maxwell problem is given by the following code:

\begin{lstlisting}[mathescape,caption=\file{MAXWELL/GALERKIN/}\routine{elem} routine]
!..determine element type
   etype = NODES(Mdle)%ntype
   
!..determine order of approximation
   call find_order(Mdle, norder)
   
!..determine edge and face orientations
   call find_orient(Mdle, norient_edge,norient_face)
   
!..determine nodes coordinates
   call nodcor(Mdle, xnod)
   
!..set quadrature points and weights
   call set_3D_int(etype,norder,norient_face, nrint,xiloc,waloc)

!  ....... element integrals:

!..loop over integration points
   do l=1,nrint

!  ...coordinates and weight of this integration point
      xi(1:3) = xiloc(1:3,l); wa = waloc(l)

!  ...H1 shape functions (for geometry)
      call shape3DH(etype,xi,norder,norient_edge,norient_face, nrdofH,shapH,gradH)

!  ...H(curl) shape functions
      call shape3DE(etype,xi,norder,norient_edge,norient_face, nrdofE,shapE,curlE)

!  ...geometry map
      call geom3D(Mdle,xi,xnod,shapH,gradH,nrdofH, x,dxdxi,dxidx,rjac,iflag)

!  ...integration weight
      weight = rjac*wa

!  ...get the RHS (complex, vector-valued)
      call getf(Mdle,x, zJ)

!  ...loop through H(curl) test functions
      do k1=1,nrdofE

!     ...Piola transformation: $F \rightarrow J^{-T} \hat F$ and $\curl F \rightarrow (\det J)^{-1} J \, \hat \nabla \times \hat F$
         F(1:3) = shapE(1,k1)*dxidx(1,1:3) &
                 + shapE(2,k1)*dxidx(2,1:3) &
                 + shapE(3,k1)*dxidx(3,1:3)
         CF(1:3) = ( dxdxi(1:3,1)*curlE(1,k1) &
                    + dxdxi(1:3,2)*curlE(2,k1) &
                    + dxdxi(1:3,3)*curlE(3,k1) ) / rjac

!     ...accumulate for the load vector: $(-i \omega J,F)$
         za = F(1)*zJ(1) + F(2)*zJ(2) + F(3)*zJ(3)
         Zbloc(k1) = Zbloc(k1) - ZI*OMEGA*za*weight

!     ...loop through H(curl) trial functions
         do k2=1,nrdofE

!        ...Piola transformation: $E \rightarrow J^{-T} \hat E$ and $\curl E \rightarrow (\det J)^{-1} J \, \hat \nabla \times \hat E$
            E(1:3) = shapE(1,k2)*dxidx(1,1:3) &
                    + shapE(2,k2)*dxidx(2,1:3) &
                    + shapE(3,k2)*dxidx(3,1:3)
            CE(1:3) = ( dxdxi(1:3,1)*curlE(1,k2) &
                       + dxdxi(1:3,2)*curlE(2,k2) &
                       + dxdxi(1:3,3)*curlE(3,k2) ) / rjac

!        ...accumulate for the stiffness matrix: $((1/\mu) \tcurl E, \tcurl F)-((\omega^2 \eps - i \omega \sigma) E, F)$
            za = (CE(1)*CF(1) + CE(2)*CF(2) + CE(3)*CF(3)) / MU
            zb = (OMEGA*OMEGA*EPS - ZI*OMEGA*SIGMA) * (E(1)*F(1) + E(2)*F(2) + E(3)*F(3))
            Zaloc(k1,k2) = Zaloc(k1,k2) + (za-zb)*weight

   enddo; enddo; enddo
\end{lstlisting}

%--------------------------------------------------------------------
\subsection{DPG ultraweak implementation}
\label{sec:maxwell-ultraweak}

The broken ultraweak Maxwell formulation is given by:
\[
\left\{
\begin{split}
	\bs E, \bs H \in \bs{L^2}(\Omega), \hat{\bs E}, \hat{\bs H} \in H^{-1/2}(\tcurl, \Gammah): \bs n \times \bs E &= \bs n \times \bs E_0 \text{ on } \Gamma \, , \\
	(\bs H, \hcurl \bs F) - ((i \omega \eps + \sigma) \bs E , \bs F)
	&= (\bs J^{\text{imp}}, \bs F) &\quad \bs F \in \hHcurl \, , \\
	(\bs E, \hcurl \bs G) + \lb \bs n \times \hat{\bs E}, \bs G \rb_{\Gammah} + (i \omega \mu \bs H, \bs G)
	&= \bs 0 &\quad \bs G \in \hHcurl \, .
\end{split}
\right .
\]

Before looking at this section, the reader is encouraged to review the simpler DPG implementations of the Poisson problem, given in Sections~\ref{sec:poisson-primal} (primal) and \ref{sec:poisson-ultraweak} (ultraweak), which are discussed in greater detail. As in the Poisson DPG implementation, the static condensation of the Riesz representation of the residual is done at the element level (cf.~Section~\ref{sec:poisson-primal}). This model problem implementation uses a scaled adjoint graph norm for the ultraweak Maxwell formulation:
\[
	\| ( \bs F, \bs G ) \|^2_\test :=
	\| \hcurl \bs F - i \omega \bar{\mu} \bs G \|^2 +
	\| \hcurl \bs G + (i \omega \bar{\eps} - \sigma) \bs F \|^2 +
	\alpha ( \| \bs F \|^2 + \| \bs G \|^2) \, ,
\]
where $\bar{\cdot}$ indicates complex-conjugate, and $\alpha > 0$ is a scaling constant. See, e.g., \cite{melenk2023waveguide1, demkowicz2023waveguide2}, for both theoretical considerations and practical implications in terms of stability of choosing a proper value for $\alpha$.

The implementation is provided in \file{problems/MAXWELL/ULTRAWEAK\_DPG}.

\begin{lstlisting}[caption=\file{MAXWELL/ULTRAWEAK\_DPG/input/physics} input file.]
100000              MAXNODS, nodes anticipated
2                   NR_PHYSA, physics attributes
EHtrc   tangen 2    traces of electric/magnetic fields, H(curl), 2 components
EHfld   discon 6    electric/magnetic fields, L2, 3+3 components
\end{lstlisting}

In the ultraweak formulation, there are four physics unknowns: $\hat{\bs E}$, $\hat{\bs H}$, $\bs E$, $\bs H$; however, since all physics variables are solved in the same formulation and the electric and magnetic fields (and their traces) each use the same respective energy spaces (namely $\bs{L^2}$ and $H^{-1/2}(\tcurl)$), one can alternatively group these variables by specifying only two physics variables with double the number of components each: $(\hat{\bs E},\hat {\bs H}) \in (H^{-1/2}(\tcurl, \Gammah))^2$ and $(\bs E, \bs H) \in (\bs{L^2}(\Omega))^2$---recall that $\bs{L^2}(\Omega) = (L^2(\Omega))^3$. The variables are defined in the \file{physics} file and the tangential traces $(\hat{\bs E},\hat {\bs H})$ are specified as such by setting \code{\var{PHYSAi(1)}=.true.}. Choosing to specify grouped physics variables in such a way has certain implications for the ordering of degrees of freedom and the allocation of element-local matrices \var{ALOC} and \var{BLOC} (see Section~\ref{sec:coupled-variables}).

The element integration routine for the ultraweak DPG formulation is given by the following code:
\begin{lstlisting}[mathescape,caption=\file{MAXWELL/ULTRAWEAK\_DPG/}\routine{elem\_maxwell}: element integration]
!  ...use the enriched order to set the quadrature
      INTEGRATION = NORD_ADD
      call set_3D_int_DPG(ntype,norder,norient_face, nrint,xiloc,waloc)

!  ...loop over integration points
      do l=1,nrint

!     ...coordinates and weight of this integration point
         xi(1:3)=xiloc(1:3,l); wa=waloc(l)

!     ...H1 shape functions (for geometry)
         call shape3DH(ntype,xi,norder,norient_edge,norient_face, nrdofH,shapH,gradH)

!     ...L2 shape functions for the trial space
         call shape3DQ(ntype,xi,norder, nrdofQ,shapQ)

!     ...broken H(curl) shape functions for the enriched test space
         call shape3EE(ntype,xi,nordP, nrdofEE,shapEE,curlEE)

!     ...geometry map
         call geom3D(Mdle,xi,xnod,shapH,gradH,nrdofH, x,dxdxi,dxidx,rjac,iflag)

!     ...get permittivity at x
         call get_permittivity(mdle,x, eps)

!     ...integration weight
         weight = rjac*wa

!     ...get the RHS
         call getf(Mdle,x, zJ)

!     ...permittivity
         za = (ZI*OMEGA*EPS) * eps(:,:)

!     ...scalar permeability
         zc1 = ZI*OMEGA*MU

!     ...apply pullbacks
         call DGEMM('T','N',3,nrdofEE,3,1.d0     ,dxidx,3,shapEE,3,0.d0,shapF,3)
         call DGEMM('N','N',3,nrdofEE,3,1.d0/rjac,dxdxi,3,curlEE,3,0.d0,curlF,3)

!     ...apply permittivity
         zshapF = cmplx(shapF,0.d0,8)
         zcurlF = cmplx(curlF,0.d0,8)
         call ZGEMM('C','N',3,nrdofEE,3,ZONE,za,3,zshapF,3,ZERO,epsTshapF,3)
         call ZGEMM('N','N',3,nrdofEE,3,ZONE,za,3,zcurlF,3,ZERO,epscurlF ,3)

!     ...loop through enriched H(curl) test functions
         do k1=1,nrdofEE

!        ...pickup pulled-back test functions
            fldF(:) = shapF(:,k1);  crlF(:) = curlF(:,k1)
            fldG(:) = fldF(:);      crlG(:) = crlF(:)
            epsTfldF(:) = epsTshapF(:,k1)

!  --- load ---
!           $(J^{\text{imp}}, F)$ first equation RHS (with first H(curl) test function F)
!           $(0, G)$ second equation RHS is zero
            n = 2*k1-1
            bload_E(n) = bload_E(n) + (fldF(1)*zJ(1)+fldF(2)*zJ(2)+fldF(3)*zJ(3)) * weight

!  --- stiffness matrix ---
!        ...loop through L2 trial shape functions
            do k2=1,nrdofQ
!           ...first L2 variable
               m = (k2-1)*6
!           ...Piola transformation
               fldE(1:3) = shapQ(k2)/rjac; fldH = fldE

!           ...$-i \omega \eps (E,F)$
!           ...$(H,\curl F)$
               n = 2*k1-1
               stiff_EQ_T(m+1:m+3,n) = stiff_EQ_T(m+1:m+3,n) - fldE(:)*conjg(epsTfldF(:))*weight
               stiff_EQ_T(m+4:m+6,n) = stiff_EQ_T(m+4:m+6,n) + fldH(:)*crlF(:)*weight

!           ...$(E, \curl G)$
!           ...$i \omega \mu (H,G)$
               n = 2*k1
               stiff_EQ_T(m+1:m+3,n) = stiff_EQ_T(m+1:m+3,n) + fldE(:)*crlG(:)*weight
               stiff_EQ_T(m+4:m+6,n) = stiff_EQ_T(m+4:m+6,n) + zc1*fldH(:)*fldG(:)*weight
            enddo !..end of loop through L2 trial functions

!  --- Gram matrix ---
!        ...loop through enriched H(curl) test functions
            do k2=k1,nrdofEE
               fldE(:) = shapF(:,k2); epsTfldE(:) = epsTshapF(:,k2)
               crlE(:) = curlF(:,k2); epscrlE(:) = epscurlF(:,k2)

               call dot_product(fldF,fldE, FF)
               call dot_product(crlF,crlE, CC)

!          ...accumulate for the Hermitian Gram matrix (compute upper triangular only)
!             ------------------------
!             | (F_i,F_j)   (F_i,G_j) |    F_i/G_i are outer loop shape functions (fldF)
!             | (G_i,F_j)   (G_i,G_j) |    F_j/G_j are inner loop shape functions (fldE)
!             ------------------------

!             (F_j,F_i) terms = Int[F_^*i F_j] terms (G_11)
               n = 2*k1-1; m = 2*k2-1; k = nk(n,m)
               zaux = conjg(epsTfldF(1))*epsTfldE(1) + &
                       conjg(epsTfldF(2))*epsTfldE(2) + &
                       conjg(epsTfldF(3))*epsTfldE(3)
               gramP(k) = gramP(k) + (zaux + ALPHA_NORM*FF + CC)*weight

!              (G_j,F_i) terms = Int[F_^*i G_j] terms (G_12)
               n = 2*k1-1; m = 2*k2; k = nk(n,m)
               zaux = -(fldF(1)*epscrlE(1) + fldF(2)*epscrlE(2) + fldF(3)*epscrlE(3))
               zcux = conjg(zc1)*(crlF(1)*fldE(1) + crlF(2)*fldE(2) + crlF(3)*fldE(3))
               gramP(k) = gramP(k) + (zaux+zcux)*weight

!           ...compute lower triangular part of 2x2 G_ij matrix
!              only if it is not a diagonal element, G_ii
               if (k1 .ne. k2) then
!                 (F_j,G_i) terms = Int[G_^*i F_j] terms (G_21)
                  n = 2*k1; m = 2*k2-1; k = nk(n,m)
                  zaux = -(crlF(1)*epsTfldE(1) + crlF(2)*epsTfldE(2) + crlF(3)*epsTfldE(3))
                  zcux = zc1*(fldF(1)*crlE(1) + fldF(2)*crlE(2) + fldF(3)*crlE(3) )
                  gramP(k) = gramP(k) + (zaux+zcux)*weight
               endif

!              (G_j,G_i) terms = Int[G_^*i G_j] terms (G_22)
               n = 2*k1; m = 2*k2; k = nk(n,m)
               zcux = abs(zc1)**2*(fldF(1)*fldE(1) + fldF(2)*fldE(2) + fldF(3)*fldE(3))
               gramP(k) = gramP(k) + (zcux + ALPHA_NORM*FF + CC)*weight
         enddo; enddo !..end of loop through enriched H(curl) test functions
      enddo !..end of loop through integration points
\end{lstlisting}

Next, we provide the code performing the boundary integration:
\begin{lstlisting}[mathescape,caption=\file{MAXWELL/ULTRAWEAK\_DPG/}\routine{elem\_maxwell}: boundary integration]
!  ...loop through element faces
      do ifc=1,nrf

!     ...sign factor to determine the outward normal unit vector
         nsign = nsign_param(ntype,ifc)

!     ...face type
         ftype = face_type(ntype,ifc)

!     ...face order of approximation
         call face_order(ntype,ifc,norder, norderf)

!     ...set 2D quadrature
         INTEGRATION = NORD_ADD
         call set_2D_int_DPG(ftype,norderf,norient_face(ifc), nrint,tloc,wtloc)

!     ...loop through integration points
         do l=1,nrint

!        ...face coordinates
            t(1:2) = tloc(1:2,l)

!        ...face parametrization
            call face_param(ntype,ifc,t, xi,dxidt)

!        ...determine discontinuous H(curl) shape functions
            call shape3EE(ntype,xi,nordP, nrdof,shapEE,curlEE)

!        ...determine element H1 shape functions (for geometry)
            call shape3DH(ntype,xi,norder,norient_edge,norient_face, nrdof,shapH,gradH)

!        ...determine element H(curl) shape functions (for fluxes) on face
            call shape3DE(ntype,xi,norderi,norient_edge,norient_face, nrdof,shapE,curlE)

!        ...geometry map
            call bgeom3D(Mdle,xi,xnod,shapH,gradH,NrdofH,dxidt,nsign, &
                          x,dxdxi,dxidx,rjac,dxdt,rn,bjac)
            weight = bjac*wtloc(l)

!        ...pullback trial and test functions
            call DGEMM('T','N',3,nrdofEE,3,1.d0,dxidx,3,shapEE,3,0.d0,shapF ,3)
            call DGEMM('T','N',3,nrdofEi,3,1.d0,dxidx,3,shapE ,3,0.d0,shapFi,3)

!        ...loop through enriched H(curl) test functions
            do k1=1,nrdofEE
               E1(1:3) = shapF(:,k1)

!           ...loop through H(curl) trial functions
               do k2=1,NrdofEi
                  E2(1:3) = shapFi(:,k2)
                  call cross_product(rn,E2, rntimesE)

                  stiff_EE_T(2*k2-1,2*k1) = stiff_EE_T(2*k2-1,2*k1) + &
                              weight*( E1(1)*rntimesE(1) + E1(2)*rntimesE(2) + E1(3)*rntimesE(3) )
               enddo !..end loop through H(curl) trial functions
            enddo !..end loop through the enriched H(curl) test functions
         enddo !..end loop through integration points
      enddo !..end loop through element faces
\end{lstlisting}

Finally, the code performing the construction of the DPG linear system:
\begin{lstlisting}[mathescape,caption=\file{MAXWELL/ULTRAWEAK\_DPG/}\routine{elem\_maxwell}: constructing DPG linear system.]
      allocate(stiff_ALL(NrTest,NrTrial+1))

!  ...Total test/trial DOFs of the element
      i1 = NrTest ; j1 = 2*NrdofEi ; j2 = 6*NrdofQ

! ...Copy stiffness and load into one matrix
      stiff_ALL(1:i1,1:j1) = transpose(stiff_EE_T(1:j1,1:i1))
      stiff_ALL(1:i1,j1+1:j1+j2) = transpose(stiff_EQ_T(1:j2,1:i1))
      stiff_ALL(1:i1,j1+j2+1) = bload_E(1:i1)

!  ...A. Compute Cholesky factorization of Gram Matrix, $\mr{G=U^* U (=LL^*)}$
      call ZPPTRF('U',NrTest,gramP,info)

!  ...B. Solve triangular system to obtain $\mr{\tilde{B}}$, $\mr{(LX=) U^* X = [B|l]}$
      call ZTPTRS('U','C','N',NrTest,NrTrial+1,gramP,stiff_ALL,NrTest,info)

      allocate(zaloc(NrTrial+1,NrTrial+1)); zaloc = ZERO

!  ...C. Matrix multiply: $\mr{B^* G^{-1} B (=\tilde{B}^* \tilde{B})}$
      call ZHERK('U','C',NrTrial+1,NrTest,ZONE,stiff_ALL,NrTest,ZERO,zaloc,NrTrial+1)

!  ...D. Fill lower triangular part of Hermitian matrix $\mr{\tilde B^* \tilde B}$
      do i=1,NrTrial
         zaloc(i+1:NrTrial+1,i) = conjg(zaloc(i,i+1:NrTrial+1))
      enddo

!..zaloc has now all blocks of the stiffness and load:
!  $z_{\text{aloc}} =  \var{ALOC(1,1)} \quad \var{ALOC(1,2)} \quad \var{BLOC(1)}$
!         $\, \var{ALOC(2,1)} \quad \var{ALOC(2,2)} \quad \var{BLOC(2)}$
\end{lstlisting}




