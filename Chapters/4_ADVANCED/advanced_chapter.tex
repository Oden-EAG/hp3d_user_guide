%
%!TEX root = ../../hp3D_user_guide.tex
%

\chapter{Advanced Topics}
\label{chap:advanced}

%--------------------------------------------------------------------

This is a preliminary version of the user manual. This chapter on the advanced FE concepts provided by the \hp3D code is currently under development and will be further expanded and completed in a future version of the user manual.

%%%%%%%%%%%%%%%%%%%%%%%%%%%%%%%
\section{Custom Boundary Conditions}
\label{sec:advanced-BC}
%%%%%%%%%%%%%%%%%%%%%%%%%%%%%%%

In the preceding chapter, it was shown how to set up basic Dirichlet boundary conditions for a variable. In this section, some custom options for setting up various types of boundary conditions are discussed. Note that the \hp3D code allows the user to customize boundary conditions in an almost arbitrary way. A few common boundary condition types are presented with the goal of introducing the general idea of customizing boundary conditions for an application.

In the user-defined \routine{set\_initial\_mesh} routine, boundary conditions are set \emph{per component:}
\begin{itemize}
	\item{
	\var{ELEMS(iel)\%bcond(1:6,1:NRINDEX)} $\in \{$ \code{0,1, ..., 9} $\}$:\\
	user-defined value indicating for faces of each initial mesh element \var{iel} whether a component is a free component (\code{0}), a Dirichlet component (\code{1}), or has a custom BC (\code{2, ..., 9}).
	} \item{
	\var{NODES(nod)\%bcond(1:NRINDEX)} $\in \{$ \code{0,1} $\}$:\\
	system-generated flag indicating for each node \var{nod} whether a component is a Dirichlet component (\code{1}) or not (\code{0}).
	}
\end{itemize}

\subsection{Neumann boundary condition}
\label{sec:impedance}

Discussion of Neumann BC will be included in a future version of the user manual.

\subsection{Impedance boundary condition}
\label{sec:impedance}

Discussion of impedance BC will be included in a future version of the user manual.

%%%%%%%%%%%%%%%%%%%%%%%%%%%%%%%
\section{Adaptivity}
\label{sec:adaptivity}
%%%%%%%%%%%%%%%%%%%%%%%%%%%%%%%

The sophisticated $hp$-adaptive capabilities are one of the main selling points of the \hp3D finite element code. This section describes the basics of an adaptive solution procedure and how the user can execute $hp$-adaptive mesh refinements in the application code.

For ultraweak DPG implementations, an effective automatic $hp$-adaptation capability is developed and described in \cite{chakraborty2023hp}. Further details on using this \hp3D adaptation module will be provided in a future version of the manual.

\subsection{Adaptive solver}
\label{sec:adaptive-solver}

The simplest adaptive algorithm executes the standard logical sequence:
\[
	\text{Solve} \longrightarrow
	\text{Estimate} \longrightarrow
	\text{Mark} \longrightarrow
	\text{Refine} 
\]
until a prescribed global error tolerance is met. Error estimation involves a loop through elements and computation of element {\em error indicators} plus possible additional information aiming at selection of an optimal element refinement. Once the element error indicators are known for all active elements, selected elements are {\em marked} for pre-selected $hp$-refinements. Two marking strategies are most popular: {\em the greedy strategy} and the {\em D\"orfler strategy}. 

In the greedy strategy, a maximum element error indicator \var{error\_max} is first determined. All elements which exceed a certain percentage of \var{error\_max} are then marked for refinement. The refinement criterion for a single element is thus
\begin{center}
	\var{error} $>$ \var{perc} * \var{error\_max}.
\end{center}

In D\"orfler's strategy \cite{dorfler1996marking}, the elements are first organized in the order of descending element error indicators and the total error \var{error\_glob} is computed by summing up the element error indicators. The first $n$ elements whose cumulative sum exceeds a certain percentage of the total error are then marked for refinement, i.e.:
\[
	\sum_{j=1}^n \var{elem\_error}_j > \var{perc} * \var{error\_glob} .
\]
Marked elements are placed on a separate list along with the requested refinement flags.

\subsection{Adaptive refinements}
\label{sec:adaptive-refinements}

This section gives an overview of the $hp$-adaptive capabilities of the \hp3D code. Note that the section does \emph{not} discuss the implementation of an error indicator function for guiding mesh adaptivity, but rather introduces the mechanism of executing the adaptive refinements. It is assumed that some error indicator function, usually problem-dependent, is provided by the user. 
% Maybe mention the DPG error indicator and examples where it's used

The starting point for adaptive refiements is an array \var{elem\_ref(:)} of size \var{nr\_elem\_ref} storing a list of the middle node numbers (\var{mdle}) corresponding to elements marked for refinement.

\paragraph{$h$-adaptivity.}
In the first example, we show how $h$-adaptive refinements are executed. In essence, each middle node is refined one-by-one by calling the \routine{refine} routine with a corresponding refinement flag \var{kref} encoding the type of $h$-refinement. For middle nodes that can be anisotropically refined, \var{kref} encodes how to refine the corresponding element. Encoding of refinement flags depends upon the element type; for instance,
\begin{itemize}
	\item a hexahedral element may be refined in 7 different ways; for example:
	\begin{itemize}
		\itemsep 0pt
		\item \code{\var{kref} = 111} : refine in $x,y,z$
		\item \code{\var{kref} = 110} : refine in $x,y$
		\item \code{\var{kref} = 101} : refine in $x,z$
	\end{itemize}
	\item a prismatic element may be refined in 3 different ways:
	\begin{itemize}
		\itemsep 0pt
		\item \code{\var{kref} = 11} : refine in $xy,z$
		\item \code{\var{kref} = 10} : refine in $xy$
		\item \code{\var{kref} = 01} : refine in $z$
	\end{itemize}
\end{itemize}

For hybrid meshes, isotropic refinements can be executed by obtaining isotropic refinement flags for any element type from the routine \routine{get\_isoref} for the respective middle node. After executing $h$-adaptive refinements, it is essential that the user calls the \routine{close\_mesh} routine to preserve 1-irregularity of the mesh.

\begin{lstlisting}[caption=Isotropic $h$-adaptive refinements., label={lst:adaptive_refinement_element_loop4}]
!..iterate over elements marked for refinement
   do iel=1,nr_elem_ref
      mdle = elem_ref(iel)
!  ...get isotropic h-refinement flag for element type
      call get_isoref(mdle, kref)
!  ...h-refine element
      call refine(mdle,kref)
   enddo
!..enforce 1-irregular mesh
   call close_mesh
!..update geometry and Dirichlet DOFs
   call update_gdof
   call update_Ddof
\end{lstlisting}

\paragraph{$p$-adaptivity.}
In the second example, we show how $p$-adaptive refinements are executed. Isotropic $p$-adaptive refinements are carried out easily by the routine \routine{execute\_pref} which must only be provided with the list of elements to be refined in the form of the corresponding middle nodes. After executing adaptive $p$-refinements, the user may want to call one of the system routines \routine{enforce\_min\_rule} or \routine{enforce\_max\_rule} to apply the minimum or maximum rule, respectively, to handle neighboring elements of different approximation order in a particular way (i.e.~either lowering or raising the order of approximation on the corresponding element interfaces).

\begin{lstlisting}[caption=Isotropic $p$-adaptive refinements., label={lst:adaptive_refinement_element_loop4}]
!..execute isotropic p-refinements for a list of elements
   call execute_pref(elem_ref,nr_elem_ref)
!..enforce minimum rule
   call enforce_min_rule
!..update geometry and Dirichlet DOFs
   call update_gdof
   call update_Ddof
\end{lstlisting}

Similar to the $h$-adaptive case where the refinement flag encoded possibly anisotropic refinements, the polynomial order $p$ of element nodes encodes the possibly anisotropic order of approximation. In order to execute anisotropic $p$-adaptive refinements, the user may instead call the routine \routine{perform\_pref}, which additionally must be provided with a list of $p$-refinement flags for middle nodes corresponding to the list of elements to be refined. For each middle node on this list, the $p$-refinement flag encodes the desired order of approximation for the middle node. For example, $p=222$ encodes isotropic quadratic order for the middle node of a hexahedral element, whereas $p=34$ encodes anisotropic mixed order for the middle node of a prism. The routine will then execute corresponding $p$-refinements of edge and face nodes. After finishing $p$-adaptive refinements, the user may again want to enforce the minimum or maximum rule.

%%%%%%%%%%%%%%%%%%%%%%%%%%%%%%%
\section{Trace Variables}
\label{sec:traces}
%%%%%%%%%%%%%%%%%%%%%%%%%%%%%%%

This section discusses how trace variables can be discretized in \hp3D. This feature is especially important for discretizations with the DPG method \cite{demkowicz2017dpg}. The \hp3D code offers conforming discretizations for traces of the exact-sequence energy spaces, i.e.~functions belonging to the $H^{1/2}$-, $H^{-1/2}(\tcurl)$-, and $H^{-1/2}$-energy spaces. In \hp3D, these trace unknowns---which are defined on element interfaces---are discretized by using restrictions of $H^1$-, $H(\tcurl)$-, and $H(\tdiv)$-conforming elements to the element boundary. The discretized trace unknowns are thus continuous, tangentially continuous, and normally continuous, respectively. For further reading, we refer to \cite{demkowicz2018spaces,demkowicz2020fem}.

To realize trace variables as restrictions of exact-sequence-conforming elements, the user specifies for each physics attribute whether it is a trace variable via a global flag:

\code{\var{PHYSAi}(i)} $\in \{$\code{.true.,.false.}$\}$, \code{i=1,...,\var{NR\_PHYSA}}.

\noindent
If enabled, this implies that interactions of bubble (interior) DOFs of an element are not stored in the element matrices for the corresponding traces. In order to compute with trace unknowns, the user \emph{must} enable \hp3D's static condensation module \module{stc} (see Section~\ref{sec:global-variables}). By default, all physics variables are defined as standard variables unless otherwise instructed by the user. The user \emph{may not} for obvious reasons request a trace of an $L^2$-conforming (discontinuous) variable.

%%%%%%%%%%%%%%%%%%%%%%%%%%%%%%%
\section{Coupled Variables}
\label{sec:coupled-variables}
%%%%%%%%%%%%%%%%%%%%%%%%%%%%%%%

In \hp3D, problems with coupled variables may be defined in a way where the coupling happens over the entire domain, over part of the domain, or at an interface between different parts of the domain. This section primarily focuses on problems with variables supported in the entire mesh.

\paragraph{Computing with multiple physics variables.}

Solving multiphysics problems in \hp3D requires an understanding of how \hp3D internally stores and computes interactions between multiple variables.
First, recall the following essential information from previous sections:
\begin{itemize}
	\item Global variables \var{NR\_PHYSA}, \var{NR\_COMP(:)}, and \var{NRINDEX} store the total number of physics attributes, the number of components per physics attribute, and the total number of components, respectively.
	\item In the \routine{set\_initial\_mesh} routine, boundary conditions are set separately for each component.
	\item In the \routine{elem} routine, element-local matrices are assembled in blocks corresponding to (supported) physics attributes:\\ 
	\var{ALOC(1:NR\_PHYSA,1:NR\_PHYSA)\%array(:,:)} (element stiffness); and \\
	\var{BLOC(1:NR\_PHYSA)\%array(:,:)} (element load).\\
	Note that the load ``vector'' may have multiple columns if solving a problem for multiple right-hand sides.
\end{itemize}

Physics variables are always defined and stored (in \module{physics}) in the order of the exact sequence. For each approximation space, the user may define multiple physics attributes (in the \file{physics} input file), and each attribute may have one or multiple components (also defined in the \file{physics} input file).

The \routine{elem} routine that assembles each of the stiffness and load blocks must provide the system assembly routines with information about which blocks have been locally computed. This is necessary for two reasons:
\begin{enumerate}
	\item An element may not support all physics variables thus will only assemble a subset of the blocks corresponding to the supported attributes.
	\item A linear system for only a subset of the supported variables is assembled and solved.
\end{enumerate}
In order to provide this information to the system routines, \routine{elem} returns a flag for each of the blocks indicating whether it was computed by the element. The following arguments are defined in the \routine{elem} routine:
\begin{itemize}
	\itemsep 0pt
	\item{ \code{integer, intent(in) :: \var{Mdle}} \\
		\var{Mdle} is the unique middle node number associated with the element.
	}
	\item{ \code{integer, intent(out) :: \var{Itest}(\var{NR\_PHYSA})} \\
		For the $i$-th test function (corresponding to the $i$-th physics attribute), \code{\var{Itest}(i) = 1} indicates that the $i$-th block (row) of \var{ALOC} and \var{BLOC} corresponds to an active and supported variable.
	}
	\item{ \code{integer, intent(out) :: \var{Itrial}(\var{NR\_PHYSA})} \\
		For the $i$-th trial function (corresponding to the $j$-th physics attribute), \code{\var{Itrial}(j) = 1} indicates that the $j$-th block (column) of \var{ALOC} corresponds to an active and supported variable.
	}
\end{itemize}

The $(i,j)$-th block of \var{ALOC} is provided by the \routine{elem} routine if and only if \code{\var{Itest}(i) = 1} \emph{and} \code{\var{Itrial}(j) = 1}, and the $i$-th block of \var{BLOC} is provided if and only if \code{\var{Itest}(i) = 1}. Usually, when the $(i,j)$-th stiffness block is computed, then the $(j,i)$-th block is also provided.

\paragraph{Solving for a subset of the physics variables.}

In a problem where the user would like to compute a subset of the variables at a time (e.g.~staggered solve in simple fixed-point iteration), this can easily be done by toggling the global flags \var{PHYSAm(:)} of the \module{physics} module. For the $i$-th physics attribute (\code{i = 1,\ldots,\var{NR\_PHYSA}}), \code{\var{PHYSAm}(i)=.false.} deactivates the corresponding physics attribute. If all physics variables are otherwise supported on the entire mesh, then the \routine{elem} routine can very easily set the assembly flags \var{Itest} and \var{Itrial} as follows:

\begin{lstlisting}[caption=Solving for a subset of variables in \routine{elem}.]
!  in:   Mdle     - element middle node number
!  out:  Itest    - index for assembly
!        Itrial   - index for assembly
subroutine elem(Mdle, Itest,Itrial)
   integer, intent(in)    :: Mdle
   integer, intent(out) :: Itest(NR_PHYSA)
   integer, intent(out) :: Itrial(NR_PHYSA)
!  .......
   Itest(1:NR_PHYSA) = 0; Itrial(1:NR_PHYSA) = 0
!
!..all variables are supported on all elements,
!  thus assemble for all of the currently enabled variables
   do i = 1,NR_PHYSA
      if (PHYSAm(i)) then
         Itest(i) = 1; Itrial(i) = 1
      endif
   enddo
!  ...... compute the corresponding blocks of ALOC and BLOC
\end{lstlisting}

\begin{remark}
More information about partial support of variables (defined in parts of the domain) will be included in a future version of the user manual.
\end{remark}

%\section*{Historical Comments}




