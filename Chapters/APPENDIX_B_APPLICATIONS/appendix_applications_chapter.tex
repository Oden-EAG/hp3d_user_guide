%
%!TEX root = ../../hp3D_user_guide.tex
%

\chapter{Applications}
\label{chap:applications}

%--------------------------------------------------------------------

To provide further examples and give the reader an idea of the scope of computations \hp3D is suitable for, this chapter briefly summarizes applications that have been implemented within the current version of the \hp3D finite element code. If an implementation is available in the public repository, we provide a file path for the interested user. References for further reading are provided in all cases.

%--------------------------------------------------------------------
\section{Optical Fiber Amplifier}
\label{sec:laser}

Fiber amplifiers are optical waveguides made of silica glass designed for power-scaling highly-coherent laser light. At high optical intensities, undesired nonlinear effects may negatively affect the beam quality. \hp3D has been used to study such nonlinear effects (e.g.~the interplay between the propagating electromagnetic fields and thermal effects) for active gain fiber amplifiers based on a finite element model of the time-harmonic Maxwell equations coupled with the heat equation \cite{henneking2021fiber,nagaraj2018raman}. An implementation of this application is available in the \file{problems/LASER} directory. 

In addition to the coupled multiphysics formulation, the application employs a high-order discretization and anisotropic adaptive refinements \cite{henneking2021pollution} for a hybrid mesh consisting of both prismatic and hexahedral elements with curvilinear geometry. The application also served as the first real testbed problem for large-scale computation with the MPI/OpenMP parallel \hp3D code, successfully scaling up to thousands of wavelengths and $\sim$1B degrees of freedom \cite{henneking2021phd,henneking2022parallel}. The DPG FE implementation of the Maxwell equations also features fast integration routines via sum factorization for both hexahedral and prismatic elements \cite{mora2019fast,badger2020fast}.

%--------------------------------------------------------------------
\section{Adaptive Solution of High-Frequency Acoustic and Electromagnetic Scattering}
\label{sec:adaptive}

Accurate and efficient numerical simulations of wave propagation phenomena are very crucial in numerous engineering and physics applications, such as non-destructive testing, plasma fusion, modeling of meta-materials and biomedical and radar imaging. With traditional discretization methods, these simulations are extremely challenging due to two major issues: instability and indefiniteness of the linear systems. Consequently, common cutting-edge elliptic solvers simply break down. The DPG method overcomes both issues. As a non-standard minimum residual method, it promises high accuracy, unconditional stability and definite linear systems.

This work implemented a novel multigrid solver within \hp3D for linear systems arising from DPG discretizations with a special focus in acoustic and electromagnetic wave propagation problems. The construction is heavily based on the attractive properties of the DPG method, but also on well-established theory of Schwarz domain decomposition and multigrid methods. As it is showcased in \cite{petrides2019phd,petrides2021adaptive,petrides2017adaptive}, the method is stable and reliable, and it is suitable for adaptive $hp$-meshes. The solver works hand-in-glove with the built-in DPG error indicator to drive adaptive mesh refinements. Integrating the iterative solver with the adaptive refinement procedure enables efficient and accurate solutions to challenging problems in the high-frequency regime. A distributed parallel version of the DPG multigrid solver is currently under development and is discussed in \cite{badger2023scalable}.

%--------------------------------------------------------------------
\section{Linear Elasticity with Two Materials}
\label{sec:hose}

Linear elasticity is a continuum model of the deformation and stresses of solids. The first finite element methods were developed to predict the effects of loads on structures modeled with linear elasticity and the study of linearly elastic materials has continued to be an important application of finite element analysis to this day. A finite element method for linear elasticity is available in the \file{problems/SHEATHED\_HOSE} directory.

The implementation employs distinct DPG methods (primal and ultraweak \cite{keith2016elasticity}) for two separate subdomains in curvilinear body. The primal DPG method is used in a subdomain with compressible material (steel) and the ultraweak DPG method is used in a subdomain with incompressible material (rubber) \cite{fuentes2017coupled}. The coupling between the two methods is naturally incorporated through the common trace and flux dofs on the element boundary. This application demonstrates support in hp3D for subdomain-dependent local element assembly, which can be used to support the solution of multiphysics problems.

%--------------------------------------------------------------------
\section{Nonlinear Elasticity}
\label{sec:nonlinear-elasticity}

Beyond linear elasticity problems, it is common in computational mechanics to model and simulate nonlinear elasticity cases, e.g.~to compute large deformations. In such cases, one needs to account for a non-convex stored energy functional that must be minimized. A practical approach for solving this minimization problem is to apply loads in incremental steps to find successive local minima via Newton--Raphson methods. Each iteration of the Newton--Raphson method applied to the system of nonlinear equations solves a linear system that is derived from a variational formulation of the linearized boundary value problem (see \cite{mora2020polydpg}, Chapters 6 and 7). 

The implementation of the nonlinear elasticity problem in \hp3D assumes an isotropic hyperelastic material with dead load undergoing quasi-static deformation. Various variational formulations of the linearized equations can be employed for this scenario and numerically solved in \hp3D with user-defined convergence tolerances. The application directory, \file{problems/NONLINEAR\_ELASTICITY}, includes several implementations: the Bubnov--Galerkin discretization of the classical variational formulation in \file{GALERKIN}; the DPG discretization of the broken primal formulation in \file{DPG\_PRIMAL}; and the DPG discretization of the broken ultraweak formulation in \file{DPG\_UW}. The mathematical theory and material constitutive models are detailed in \cite{mora2020polydpg}. A more thorough description of the implementation for this problem is presented in  \cite{hpbook3}.

%--------------------------------------------------------------------
\section{Time-Harmonic Applications in Linear Viscoelasticity}
\label{sec:visco-elasticity}

The linearized equations describing the dynamics of viscoelastic material behavior in the time-harmonic regime are very similar to those of time-harmonic linear elasticity. The difference is that the material properties themselves (as well as the relevant unknowns) are complex-valued. The reason is that the (linearized) constitutive model describing the relationship between stress and strain depends on the strain history, and is given by a Volterra-type convolution, which becomes a product in the time-harmonic regime. This allows to model real-life experiments in dynamic mechanical analysis (DMA) as well as phenomena that involve low-amplitude vibrations of viscoelastic material at different frequencies.

An example where \hp3D was used includes the faithful reproduction of a DMA experiment involving a thin beam. Here support for complex variables is needed, as well as anisotropic $p$ and local anisotropic refinements in $h$ when the domain is discretized with hexahedra. The reason is that adaptivity is localized in certain regions where most of the deformation gradient is concentrated due to nontrivial clamping effects \cite{fuentes2017viscoelasticity}. Moreover, in an application involving form-wound stator coils found in electric machinery, it is necessary to introduce a nontrivial ``surface'' forcing coming from high-frequency Lorentz forces. This type of forcing is not typically supported. However, with the proper variational formulation and boundary conditions, it is possible to use \hp3D to solve this problem, which also includes some of the features mentioned previously \cite{fuentes2018phd}.

%--------------------------------------------------------------------
\section{Acoustics of the Human Head}
\label{sec:human-head}

The problem was formulated and solved in the course of the dissertation of Paolo Gatto devoted to the analysis of transmission of sound in the human head \cite{gatto2012phd}, see also \cite{demkowicz2011bone,gatto2013bone}. More specifically, it studied the mechanism of transmitting exterior acoustic energy into the cochlea with the ear canal {\em blocked}. The domain of interest included the human skull modeled with a spherical shell with an inner structure representing the ear canal, a cochlea with the tympanic membrane, and a simplified model of ossicles consisting of a single elastic rod connecting the ear drum with the oval window in the cochlea. The skull was surrounded with an additional layer representing air, terminated with a PML.
The skull, ear canal, cochlea, ossicles and the three membranes were modeled with elasticity; whereas the air, exterior and interior ear, the brain and the fluid inside of the cochlea were modeled with acoustics, a total of 16 subdomains whose geometry was represented within the GMP package. The initial tetrahedral linear geometry/mesh was obtained using NETGEN and imported into GMP where two important modifications were made. The skull and all membranes, represented initially with zero thickness interfaces, were extruded into thin-walled structures, and the triangular mesh representing the skull was extruded into two additional layers for air and PML. The extrusions brought in both prismatic and pyramidal elements which motivated adding these elements to the \hp3D code.

In the elastic subdomains, the code supported the elastic displacement field---a single $H^1$ field with three components. In the acoustical subdomains, the code supported the pressure---a single-component $H^1$ field. Note that the nodes located on acoustic/elastic interfaces supported both variables. More precisely, in the external acoustic domain the acoustic variable represented the scattered pressure, and in the internal acoustical domains the total pressure. The forcing came from a plane wave impinging on the skull, resulting with an additional term on the exterior acoustics/elasticity interface. A standard variational formulation (see e.g.~\cite{hpbook2}) was used with couplings between the elasticity and acoustic problems enforced weakly by incorporating interface integrals in the variational formulation. 
%It is worth mentioning that the (asymptotic) discrete stability of the problem is guaranteed by the Mikhlin theory \cite{demkowicz2020fem}.
The problem was discretized with the standard Galerkin method. In the initial mesh, the three membranes were discretized with quintic elements to avoid locking, with the rest of the domain covered with quadratic elements except for the PML subdomain where higher order anisotropic elements were used. Original shape functions for $H^1$-conforming elements of all shapes were developed in \cite{gatto2010shape}. Finally, the problem was solved using $h$-adaptivity driven by a problem-dependent, implicit a-posteriori error estimate developed in the course of the project.

%--------------------------------------------------------------------
\section{Electromagnetic Radiation and Induced Heat Transfer in the Human Body}
\label{sec:human-body}

The project focused on estimating heating effects in the human head from the absorption of electromagnetic (EM) waves emanating from a  cell phone; it constituted (partially) the dissertation of Kyungjoo Kim \cite{Kim2013phd}.

The transient heat equation was coupled with time-harmonic Maxwell equations; the standard Galerkin method was used for both heat and Maxwell problem. The heat equation was discretized with the implicit Euler method. At each time step, the new temperature field was used to compute new (temperature-dependent) material constants for the Maxwell equations. Due to disparate time scales, the time-harmonic version of the Maxwell equations was employed. The resulting new electric field was then used to update the source in the heat equation. The iterations were continued until a steady-state was reached.

The head was modeled with both simplified multilayer spherical geometry as well as a more precise (homogeneous) head model imported from COMSOL. The application was the second problem driving the original development of the current version of the \hp3D code. The Maxwell problem was discretized with $H(\tcurl)$-conforming tetrahedral and prismatic elements.

\clearpage

%\section*{Historical Comments}




